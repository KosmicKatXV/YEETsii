\documentclass[english, spanish, fleqn, 10pt]{article}
\usepackage[top = 2.5cm, bottom = 2cm, left = 2.5cm, right = 2.5cm]{geometry}
\usepackage{babel}
\usepackage[utf8]{inputenc}
\usepackage{amsthm, amsmath, amssymb, amsfonts, environ}
\usepackage{nccmath} %Para centrar ecuaciones
\usepackage{caption}
\usepackage{mathrsfs}
\usepackage[hidelinks]{hyperref}
%\usepackage{fourier}
\usepackage{colortbl}
\usepackage{color}
\usepackage{graphicx}
\usepackage{enumitem}
%\renewcommand{\familydefault}{\sfdefault}
%\setlength{\parindent}{0pt}					%Espacio de sangria
\setlength{\parskip}{2mm}						%Interlinado entre párrafos
\usepackage{fancyhdr, blindtext}		%Paquete de encabezado
\usepackage{listings}
\usepackage[framemethod=tikz]{mdframed}

\definecolor{webred}{rgb}{0.75,0,0}
\definecolor{mblue}{rgb}{0.0705,0.345,0.7529}

%==============================
%Modificador de Titulos de secciones, subsecciones, etc
\RequirePackage{titlesec}

\titleformat{\section}
{\normalfont\Large\bfseries}{\thesection.}{1em}{}[\titlerule]

\titleformat{\subsubsection}
{\bfseries\raggedright}{\thesubsubsection.}{.4em}{}
\titlespacing\subsubsection{0pt}{8pt plus 4pt minus 2pt}{0pt plus 2pt minus 2pt}

%==================================
%Diseno para insertar codigos de programacion
\definecolor{newGray}{RGB}{240, 240, 240}
\definecolor{anotherGray}{HTML}{BFBFBF}
\lstdefinestyle{customc}{
	belowcaptionskip=1\baselineskip,
	breaklines=true,
	backgroundcolor=\color{newGray},
	frame=single,
	rulecolor=\color{anotherGray},
	xleftmargin=\parindent,
	language=bash,
	showstringspaces=false,
	basicstyle=\ttfamily\small,
	keywordstyle=\bfseries\color{green!40!black},
	commentstyle=\itshape\color{purple!40!black},
	identifierstyle=\color{black},
	stringstyle=\color{mblue},
	tabsize=2,
	literate={\ \ }{{\ }}1	
}
\lstset{style=customc}

%==================================
%Macros útiles
\newcommand{\comillas}[1]{``#1''}
\newcommand{\comentarioc}[1]{\texttt{\textcolor{webred}{/* #1 */}}}
\newcommand{\definicion}[1]{\textit{\underline{\smash{#1}}}}

\numberwithin{equation}{section}

%=================================
%comandos matemáticos personalizados de rápido acceso
\newcommand{\nparentesis}[1]{\left( #1 \right)}
\newcommand{\nint}[3]{\int_{#1}^{#2}{#3}}
\newcommand{\nsum}[3]{\sum_{#1}^{#2}{#3}}
\newcommand{\llaves}[1]{\left \{ #1 \right \}}
\newcommand{\nabsoluto}[1]{\left| #1 \right|}
\newcommand{\ncorchetes}[1]{\left[ #1 \right]}
%=================================

%cambia el espaciado entre el parrafo y antes del itemize
\setlist[itemize]{itemsep=1mm, topsep=0.5mm}
\setlist[enumerate]{itemsep=1mm, topsep=0.5mm}
%===========================

% \theoremstyle{definition}
% \newtheorem{teorema}{Teorema}[section]
% \newtheorem{definition}{Definición}[section]
% \newtheorem{preposicion}{Proposición}[section]

% Personalizo mi alfabeto
\DeclareMathAlphabet{\pazocal}{OMS}{zplm}{m}{n}
\newcommand{\Lb}{\pazocal{L}}

\theoremstyle{plain}
\newtheorem{theorem}{Teorema}[section]
\newtheorem{lema}[theorem]{Lema}
\newtheorem{proposicion}[theorem]{Proposición}
\newtheorem*{corolario}{Corolario}

\theoremstyle{definition}
\newtheorem{definition}{Definición}[section]
\newtheorem{conj}{Conjetura}[section]
\newtheorem{example}{Ejemplo}[section]

\theoremstyle{remark}
\newtheorem*{obs}{Observación}
\newtheorem*{nota}{Nota}

\usepackage{fancyhdr}

\pagestyle{fancy}
\lhead{Teoría de Autómatas y Lenguajes Formales}
\rhead{\textit{Práctica 1\quad Latex y expresiones regulares}}
\renewcommand{\headrulewidth}{2pt}

\newcommand{\OR}{\, \big |\,}

\title{Teoría de Autómatas y Lenguajes Formales\\[.4\baselineskip]Práctica 1: Latex y expresiones regurales}
\author{Nombre, Apellidos}
\date{\today}

%==========================

\begin{document}

\maketitle

\section{Expresiones regulares}

Las \textit{expresiones regulares} ($\mathcal{R}$) son un método de representación de
lenguajes. Aunque su potencia expresiva es limitada, haciendo que sólo los
lenguajes regulares puedan representarse con ellas, tienen la virtud de una gran
sencillez en su formulación.

\subsection{Definiciones}

\begin{definition}
	Dado un alfabeto $\Sigma$ definimos las \definicion{expresiones regulares} sobre $\Sigma$ como las cadenas sobre el alfabeto $\Sigma'=\Sigma\cup\{),(,\emptyset,+,^*\}$ que pueden construirse con las siguientes reglas:
%\renewcommand{\labelenumi}{\alph{enumi})}
  \begin{enumerate}[label=\alph{enumi})]
    \item $\emptyset$ es $\mathcal{R}$. 
    \item Si $\alpha,\beta$ son $\mathcal{R}$ entonces $(\alpha\beta)$ es $\mathcal{R}$.
    \item Si $\alpha,\beta$ son $\mathcal{R}$ entonces $(\alpha+\beta)$ es $\mathcal{R}$.
    \item Si $\alpha$ son $\mathcal{R}$ entonces $\alpha^*$ es $\mathcal{R}$.
    \item Ninguna cadena sobre $\Sigma'$ que no se pueda construir con las reglas anteriores es $\mathcal{R}$.
  \end{enumerate}
\end{definition}

\medskip

\begin{definition}[\textbf{\textit{Aplicación $\Lb$}}]\label{def:aplicL}
	La \definicion{aplicación $\Lb$} establece una relación formal entre las expresiones regulares y los lenguajes que éstos representan, definiéndose como sigue:
  \begin{ceqn}	%Para definir ecuación centrada en el texto
    \begin{align*} %Ecuación multilínea con alineamiento personalizado (split y align)
    \Lb: &\mathcal{R} \rightarrow 2^{\Sigma^*}\\ 
    & \mathcal{R} \rightarrow \Lb(\mathcal{R})
    \end{align*} 
  \end{ceqn} 
%\renewcommand{\labelenumi}{\alph{enumi})}
\begin{enumerate}[label=\alph{enumi})]
  \item $\Lb(\emptyset)=\emptyset$ 
  \item $\Lb(a)=\llaves{a} \, \forall a\in\Sigma$ 
  \item Si $\alpha,\beta \in \mathcal{R}$ entonces $\Lb((\alpha\beta))=\Lb(\alpha)\Lb(\beta)$
  \item Si $\alpha,\beta \in \mathcal{R}$ entonces $\Lb((\alpha+\beta))=\Lb(\alpha)\cup \Lb(\beta)$
  \item Si $\alpha \in \mathcal{R}$ entonces $\Lb(\alpha^*)=\Lb(\alpha)^*$
  
\end{enumerate}
\end{definition}

\subsection{Propiedades de las expresiones regulares}
\begin{proposicion}
Si $\alpha,\beta,\gamma$ son expresiones regulares entonces se cumple:
  \begin{equation}
  (\alpha+\beta)\gamma=\alpha\gamma+\beta\gamma
  \end{equation}
\end{proposicion}


\end{document}
